
An overview of the Nios II processor can be found in the document {\it Introduction to the
Nios~II Processor}, which is available in the
\texttt{Computer Organization System Design} section of the 
{\small \href{https://www.fpgacademy.org/tutorials.html} {FPGAcademy.org}} website.
An easy way to begin working with the \systemName~and the Nios~II
processor is to make use of a utility called the {\it \productNameMed{}}.  It
provides an easy way to assemble/compile Nios II programs 
written in either assembly language or the C language. The Monitor 
Program, which can be downloaded from the
{\small \href{https://www.fpgacademy.org/tools.html} {FPGAcademy.org}} website,
is an application program that 
runs on the host computer connected to the \DEBoard~board.  The Monitor Program can be 
used to control
the execution of code on Nios~II, list (and edit) the contents of processor registers, 
display/edit the contents of memory on the \DEBoard~board, and similar operations.
The Monitor Program includes the \systemName~as a predesigned system that can be
downloaded onto the \DEBoard~board, as well as several sample programs in assembly language and
C that show how to use the \systemName's peripherals.
Some images that show how the \systemName~is integrated with the 
Monitor Program are described in Section~\ref{sec:monitor_program}.

All of the I/O peripherals in the \systemName~are
accessible by the processor as memory mapped devices, 
using the address ranges that are given in the following subsections.

