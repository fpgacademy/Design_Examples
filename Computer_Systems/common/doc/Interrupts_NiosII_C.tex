\subsection{Using Interrupts with C Language Code}

An example of C language code for the \systemName~that uses interrupts is
shown in Listing \ref{lst:interrupt_example_C}. This code performs exactly the same
operations as the code described in Listing \ref{lst:interrupt_example_s}. 

To enable interrupts the code in Listing \ref{lst:interrupt_example_C} uses {\it macros} 
that provide access to the Nios II status 
and control registers.  A collection of such macros, which can be used in any C program,
are provided in Listing \ref{lst:macros}. 

The reset and exception handlers for the main program in Listing \ref{lst:interrupt_example_C} 
are given in Listing \ref{lst:exception_handler_C}. 
The function called {\it the\_reset} provides a simple reset mechanism by
performing a branch to the main program. The function named {\it the\_exception} 
represents a general exception handler that can be used with any C program. It includes 
assembly language code to check if the exception is caused by an external interrupt, and, 
if so, calls a C language routine named {\it interrupt\_handler}. This routine can then 
perform whatever action is needed for the specific application. In 
Listing~\ref{lst:exception_handler_C}, the {\it interrupt\_handler} code first 
determines which exception has occurred, by using a macro from Listing~\ref{lst:macros} 
that reads the content of the Nios II interrupt pending register.  The interrupt service 
routine that is invoked for the interval timer is shown in \ref{lst:interval_timer_isr_C}, 
and the interrupt service routine for the pushbutton switches appears in 
Listing~\ref{lst:pushbutton_isr_C}.

The source code files shown in Listing \ref{lst:interrupt_example_s} to
Listing~\ref{lst:pushbutton_isr_C} are distributed as part of the  
\productNameMed{}. The files can be found under the heading {\it sample programs}, 
and are identified by the name {\it Interrupt Example}.



