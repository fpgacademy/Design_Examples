\subsubsection{Using the Parallel Ports with Assembly Language Code and C Code}

The \systemName~provides a convenient platform for experimenting with \processor~
assembly language code, or C code.  A simple example of such code is provided in 
Figures \ref{fig:getting_started_s} and \ref{fig:getting_started_C}. Both programs
perform the same operations, and illustrate the use of parallel ports by using either 
assembly language or C code.

The code in the figures displays the values of the SW switches on the red lights {\it LEDR}.
It also displays a rotating pattern on 7-segment displays {\it HEX3}, $\ldots$, {\it HEX0}. 
This pattern is rotated to the left by using \aProcessor~{\it rotate} instruction, and a delay loop 
is used to make the shifting slow enough to observe. The pattern on the HEX displays
can be changed to the values of the SW switches by pressing any of 
pushbuttons {\it KEY}$_{3-0}$.
When a pushbutton key is pressed, the program waits in a loop until the key is released.

The source code files shown in Figures \ref{fig:getting_started_s} and \ref{fig:getting_started_C}
are distributed as part of the  
\productNameMed{}. The files can be found under the heading {\it sample programs}, 
and are identified by the name {\it Getting Started}.

\newpage
\begin{figure}[h!]
\begin{center}
\begin{minipage}[t]{12.5 cm}
\begin{tabbing}
/\=*****\=*********************************\=****************************************\=\kill
/********************************************************************************\\
\>* This program demonstrates the use of parallel ports in the \systemName:\\
\>* \>1. displays the SW switch values on the red LEDR\\
\>* \>2. displays a rotating pattern on the HEX displays\\
\>* \>3. if any KEY is pressed, the SW switches are used as the rotating pattern\\
\=\kill
\>********************************************************************************/\\
ZZZZZZZZ\={\bf movia}ZZZ\=r16, 0xFF200000ZZZZZZZZ\=/* RED\_LED base address */\kill
\>.{\bf text}	\>\>/* executable code follows */\\
\>.{\bf global} \>\_start\\
\_start:\\
\>/* initialize base addresses of parallel ports */\\
\>{\bf movia} \>r15, 0xFF200040 \>/* SW slider switch base address */\\
\>{\bf movia} \>r16, 0xFF200000 \>/* red LED base address */\\
\>{\bf movia} \>r17, 0xFF200050 \>/* pushbutton KEY base address */\\
\>{\bf movia} \>r20, 0xFF200020 \>/* HEX3\_HEX0 base address */\\
\>{\bf movia} \>r19, HEX\_bits\\
\>{\bf ldwio} \>r6, 0(r19) \>/* load pattern for HEX displays */\\
\rule{6.0in}{0in}~\\
DO\_DISPLAY:\\
\>{\bf ldwio} \>r4, 0(r15) \>/* load input from slider switches */\\
\>{\bf stwio} \>r4, 0(r16) \>/* write to red LEDs */\\
\>{\bf ldwio} \>r5, 0(r17) \>/* load input from pushbuttons */\\
\>{\bf beq} \>r5, r0, NO\_BUTTON\\
\>{\bf mov} \>r6, r4 \>/* copy SW switch values onto HEX displays */\\
WAIT:\\
\>{\bf ldwio} \>r5, 0(r17) \>/* load input from pushbuttons */\\
\>{\bf bne} \>r5, r0, WAIT \>/* wait for button release */\\
NO\_BUTTON:\\
\>{\bf stwio} \>r6, 0(r20) \>/* store to HEX3 ... HEX0 */\\
\>{\bf roli} \>r6, r6, 1 \>/* rotate the displayed pattern */\\
\>{\bf movia} \>r7, 500000 \>/* delay counter */\\
DELAY:	\\
\>{\bf subi} \>r7, r7, 1\\
\>{\bf bne} \>r7, r0, DELAY	\\
\>{\bf br} \>DO\_DISPLAY\\
~\\
\>.{\bf data}	\>\>/* data follows */\\
HEX\_bits:\\
\>.{\bf word} \>0x0000000F\\
\>.{\bf end}\\
\end{tabbing}
\end{minipage}
\end{center}
	\vspace{-0.33in}\caption{An example of assembly language code that uses parallel ports.}
   \label{fig:getting_started_s}
\end{figure}
\pagebreak
\clearpage
\newpage

\newpage
\begin{figure}[h!]
\begin{center}
\begin{minipage}[t]{12.5 cm}
\begin{tabbing}
/\=*****\=*********************************\=****************************************\=\kill
/********************************************************************************\\
\>* This program demonstrates the use of parallel ports in the \systemName\\
\>* It performs the following: \\
\>* \>1. displays the SW switch values on the red lights LEDR\\
\>* \>2. displays a rotating pattern on the HEX displays\\
\>* \>3. if any KEY is pressed, the SW switches are used as the rotating pattern\\
\=\kill
\>********************************************************************************/\\
{\bf int} main({\bf void})\\
\{\\
ZZ\=/\=* \=\kill
\>/* Declare volatile pointers to I/O registers (volatile means that the locations will
not be cached,\\
\>\>* even in registers) */\\
ZZ\={\bf volatile} {\bf int} * HEX3\_HEX0\_ptr	Z\== ({\bf int} *) 0xFF200000;ZZZ\=\kill
\>{\bf volatile} {\bf int} * LED\_ptr \>= ({\bf int} *) 0xFF200000; \>// red LED address\\
\>{\bf volatile} {\bf int} * HEX3\_HEX0\_ptr	\>= ({\bf int} *) 0xFF200020; \>// HEX3\_HEX0 address\\
\>{\bf volatile} {\bf int} * SW\_switch\_ptr	\>= ({\bf int} *) 0xFF200040; \>// SW slider switch address\\
\>{\bf volatile} {\bf int} * KEY\_ptr \>= ({\bf int} *) 0xFF200050;	\>// pushbutton KEY address\\
\rule{6.0in}{0in}~\\
ZZ\=ZZZ\=ZZZ\=HEX\_bits = HEX\_bits z 0xFFFFFFFF;ZZ\=// initial pattern for HEX displays\kill
\>{\bf int} HEX\_bits = 0x0000000F; \>\>\>// initial pattern for HEX displays\\
\>{\bf int} SW\_value;\\
\>{\bf volatile int} delay\_count;\>\>\>// volatile so C compiler does not remove loop\\
~\\
\>{\bf while} (1)\\
\>\{\\
\>\>SW\_value = *(SW\_switch\_ptr); \>\>// read the SW slider switch values\\
\>\>*(LED\_ptr) = SW\_value; \>\>// light up the red LEDs\\
~\\
\>\>{\bf if} (*KEY\_ptr != 0) \>\>// check if any KEY was pressed\\
\>\>\{\\
\>\>\>HEX\_bits = SW\_value; \>// set pattern using SW values\\
\>\>\>{\bf while} (*KEY\_ptr != 0); \>// wait for pushbutton KEY release\\
\>\>\}\\
\>\>*(HEX3\_HEX0\_ptr) = HEX\_bits; \>\>// display pattern on HEX3 ... HEX0\\
~\\
\>\>/* rotate the pattern shown on the HEX displays */\\
\>\>{\bf if} (HEX\_bits \& 0x80000000)\\
\>\>\>HEX\_bits = (HEX\_bits $<<$ 1) $\mid$ 1;\\
\>\>{\bf else}\\
\>\>\>HEX\_bits = HEX\_bits $<<$ 1;\\
~\\
\>\>{\bf for} (delay\_count = 100000; delay\_count != 0; $-\,-$delay\_count);\hspace{.75
cm}// delay loop\\
\>\} \\
\}\\
\end{tabbing}
\end{minipage}
\end{center}
	\vspace{-0.33in}\caption{An example of C code that uses parallel ports.}
   \label{fig:getting_started_C}
\end{figure}
\pagebreak
\clearpage
\newpage