\subsection{Interrupts}

% Assembly source files

% interrupt example
\begin{center} \begin{minipage}[h]{16.5 cm}
\lstinputlisting[style=defaultNiosVStyle, name=interrupt_example_s, caption={An example of assembly language code
that uses interrupts (Part $a$).}, captionpos=b, label={lst:interrupt_example_s}, lastline=43,
showlines=true, numbers=left]{\sampleProgramsPath/niosv/asm/interrupt_example/interrupt_example.s}
\end{minipage} \end{center}

\newpage
\begin{center} \begin{minipage}[h]{15 cm}
\expandparam
\lstinputlisting[style=\processorStyle, name=interrupt_example_s, firstnumber=last,
captionpos=b, firstline=44, lastline=90, showlines=true,
numbers=left]{\sampleProgramsPath/\processorLower/asm/interrupt_example/interrupt_example.s}
\end{minipage} \end{center}
\begin{center}
Listing \ref{lst:interrupt_example_s}. An example of assembly language code that uses
interrupts (Part {\it b}).
\end{center}

\newpage
\begin{center} \begin{minipage}[h]{15 cm}
\expandparam
\lstinputlisting[style=\processorStyle, name=interrupt_example_s, firstnumber=last,
captionpos=b, firstline=91, lastline=130, showlines=true, 
numbers=left]{\sampleProgramsPath/\processorLower/asm/interrupt_example/interrupt_example.s}
\end{minipage} \end{center}
\begin{center}
Listing \ref{lst:interrupt_example_s}. An example of assembly language code that uses
interrupts (Part {\it c}).
\end{center}

\newpage
\begin{center} \begin{minipage}[h]{16 cm}
\expandparam
\lstinputlisting[style=\processorStyle, name=interrupt_example_s, firstnumber=last,
captionpos=b, firstline=131, lastline=176, showlines=true, numbers=left]{\sampleProgramsPath/\processorLower/asm/interrupt_example/interrupt_example.s}
\end{minipage} \end{center}
\begin{center}
Listing \ref{lst:interrupt_example_s}. An example of assembly language code that uses
interrupts (Part {\it d}).
\end{center}

\newpage
\begin{center} \begin{minipage}[h]{16 cm}
\expandparam
\lstinputlisting[style=\processorStyle, name=interrupt_example_s, firstnumber=last,
captionpos=b, firstline=177, showlines=true, numbers=left]{\sampleProgramsPath/\processorLower/asm/interrupt_example/interrupt_example.s}
\end{minipage} \end{center}
\begin{center}
Listing \ref{lst:interrupt_example_s}. An example of assembly language code that uses
interrupts (Part {\it e}).
\end{center}

\newpage
% C Source Code

\begin{center} \begin{minipage}[h]{16.5 cm}
\lstinputlisting[language=C, caption={An example of C code that uses interrupts (Part $a$).},
captionpos=b, label={lst:interrupt_example_C}, numbers=left, lastline=48, showlines=true]{\sampleProgramsPath/niosv/c/interrupt_example/interrupt_example.c}
\end{minipage} \end{center}
\newpage

\begin{center} \begin{minipage}[h]{15 cm}
\lstinputlisting[language=C, numbers=left, firstnumber=last, firstline=49, lastline=95,
showlines=true]{\sampleProgramsPath/niosv/c/interrupt_example/interrupt_example.c}
\end{minipage} \end{center}
\begin{center}
Listing \ref{lst:interrupt_example_C}. An example of C code that uses interrupts (Part {\it b}).
\end{center}
\newpage

\begin{center} \begin{minipage}[h]{15 cm}
\lstinputlisting[language=C, numbers=left, firstnumber=last, firstline=96, lastline=136,
showlines=true]{\sampleProgramsPath/niosv/c/interrupt_example/interrupt_example.c}
\end{minipage} \end{center}
\begin{center}
Listing \ref{lst:interrupt_example_C}. An example of C code that uses interrupts (Part {\it c}).
\end{center}
\newpage

\begin{center} \begin{minipage}[h]{15 cm}
\lstinputlisting[language=C, numbers=left, firstnumber=last, firstline=137,
showlines=true]{\sampleProgramsPath/niosv/c/interrupt_example/interrupt_example.c}
\end{minipage} \end{center}
\begin{center}
Listing \ref{lst:interrupt_example_C}. An example of C code that uses interrupts (Part {\it d}).
\end{center}
\newpage
