\subsection{Interrupts from the FPGA Interval Timer}

Figure \ref{fig:interval_port}, in Section \ref{sec:interval_port}, shows six registers that
are associated with the interval timer. As we said in Section \ref{sec:interval_port}, the
{\it TO}~bit in the {\it Status} register is set to 1 when the timer reaches a count value of 0.
It is possible to generate an interrupt when this occurs, by using the {\it ITO}~bit in 
the {\it Control} register. Setting the {\it ITO}~bit to 1 causes an interrupt request to 
be sent to the \GIC~whenever {\it TO} becomes 1. After an interrupt occurs, it can be cleared 
by writing any value into the {\it Status} register.


