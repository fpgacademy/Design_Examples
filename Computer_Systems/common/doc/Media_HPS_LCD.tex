\subsection{LCD Port}

The \systemName~includes a liquid crystal display (LCD) port that is connected to the 128 $\times$ 64 pixel display on the \DEBoard~board. Data is written to the LCD in pages where each page consists of an 128 $\times$ 8 pixel section of the display for a total of 8 pages. Each column of a page is presented as a byte, where each bit corresponds to one pixel in that column. To write to the display, the page address is first written to the data register of SPIM0 followed by the column addresses. Then the corresponding 8-bits of pixel data may be written. After writing one 8-bit column, the column address is incremented automatically, so that consecutive writes to the data register will be for adjacent columns. \\
~\\
Before writing to the display, SPIM0 and the LCD need to be initialized. To do this, SPIM0, which has address {\sf 0xFFF00000} is taken out of reset and put into transfer only mode. Then the baud rate is set to 0x40 and the slave register is enabled. Interrupts should be turned off before SPIM0 is turned on. To initialize the LCD, the output direction register of GPIO1, which has address {\sf 0xFF709000} is set to point to the LCD. Then the LCD backlight can be turned on and taken out of reset. Finally, the LCD registers must be initialized.
An example using the LCD is provided in the Appendix at Listing \ref{lst:LCD_example_C}.