\subsubsection{Using the Parallel Ports with Assembly Language Code and C Code}

The {\it \systemNameFull} provides a convenient platform for experimenting with {\processor}
assembly language code, or C code.  A simple example of such code is provided in the Appendix in
Listings \ref{lst:getting_started_s} and \ref{lst:getting_started_C}. Each of these
listing {\it includes} a file that specifies the memory-mapped addresses of all peripheral
devices in the {\it \systemNameFull}. These include files, called {\it address\_map\_niosV.s}
and {\it address\_map\_niosV.h}, are provided in Listings~\ref{lst:address_map_niosV_s}
and~\ref{lst:address_map_niosV_h}. These include files are also used in other code
samples described in this document.  

The code in Listing \ref{lst:getting_started_s} and \ref{lst:getting_started_C}
displays the values of the SW switches on the LED lights, and also
shows a rotating pattern on the LEDs. This pattern is shifted in a loop, using a software delay
to make the shifting slow enough to observe. The pattern can be changed to the values of the 
SW switches by pressing a pushbutton KEY. When a KEY is pressed, the program waits in a
loop until it is released and then continues to display the pattern.

The source code files shown in Listings \ref{lst:getting_started_s} and \ref{lst:getting_started_C}
are distributed as part of the  
\productNameMed{}. The files can be found under the heading {\it sample programs}, 
and are identified by the name {\it Getting Started}.


