\section{Exceptions and Interrupts}
\label{sec:exceptions}

The A9 processor supports eight types of exceptions, including the {\it reset} exception and the
{\it interrupt request} (IRQ) exception, as well a number of exceptions related to 
error conditions. All of the exception types are described in the document 
{\it Introduction to the ARM Cortex-A9 Processor}, which is provided in 
the \texttt{Computer Organization System Design} section of the 
\href{https://www.fpgacademy.org/tutorials.html} {FPGAcademy.org} website.
Exception processing uses a table in memory, called the {\it vector table}. This 
table comprises eight words in memory and has one entry for each type of exception. The 
contents of the vector table have to be set up by software, which typically places a 
branch instruction in each word of the table, where the branch target is the desired 
exception service routine. When an exception occurs, the ARM processor stops the 
program that is currently running, and then fetches the instruction 
stored at the corresponding vector table entry.  The vector table usually starts at 
the address {\sf 0x00000000} in memory. The first entry in the table corresponds to the 
reset vector, and the IRQ vector uses the seventh entry in the table, at the 
address {\sf 0x00000018}.

The IRQ exception allows I/O peripherals to generate interrupts for the A9 processor.
All interrupt signals from the peripherals are connected to a module in the processor
called the {\it Generic Interrupt Controller} (GIC). The GIC allows individual interrupts
for each peripheral to be either enabled or disabled. When an enabled interrupt happens, the 
GIC causes an IRQ exception in the A9 processor. Since the same vector table entry is used for 
all interrupts, the software for the interrupt service routine must determine the source 
of the interrupt by querying the GIC. Each peripheral is identified in the GIC by an
interrupt identification (ID) number.  Table \ref{tab:irq}
gives the assignment of interrupt IDs for each of the I/O peripherals in the \systemName.
The rest of this section describes the interrupt behavior associated with the 
timers and parallel ports.

