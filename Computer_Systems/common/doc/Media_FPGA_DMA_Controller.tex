\subsubsection{DMA Controller for Video}
\label{sec:dma_video}

The data provided by the Video-In core is stored into memory using a DMA Controller for Video. When
operating in {\it Stream to Memory} mode, the DMA stores the incoming frames to memory. 
Table~\ref{tab:video_dma} describes the registers used in the DMA Controller.

\begin{table}[h]
    \centering
    \begin{tabular}{|c|l|c|c|c|c|c|c|c|c|c|c|}
        \hline
            \textbf{Address}
            & \multicolumn{1}{c|}{\textbf{Register}}
            & \multirow{2}{*}{\textbf{R/W}}
            & \multicolumn{9}{c|}{\textbf{Bit Description}}
        \\\cline{4-12}
            & \multicolumn{1}{c|}{\textbf{Name}}
            &
            & \textbf{31\ldots24}
            & \textbf{23\ldots16}
            & \textbf{15\ldots12}
            & \textbf{11\ldots8}
            & \textbf{7\ldots6}
            & \textbf{5\ldots3}
            & \textbf{2}
            & \textbf{1}
            & \textbf{0}
        \\\hline
            0xFF203060
            & \texttt{Buffer}
            & R
            & \multicolumn{9}{c|}{Buffer's start address}
        \\\hline
            0xFF203064
            & \texttt{BackBuffer}
            & R/W
            & \multicolumn{9}{c|}{Back buffer's start address}
        \\\hline
            0xFF203068
            & \texttt{Resolution}
            & R
            & \multicolumn{2}{c|}{Y}
            & \multicolumn{7}{c|}{X}
        \\\hline
            \multirow{2}{*}{0xFF20306C}
            & \texttt{Status}
            & R
            & m
            & n
            & {\footnotesize \it (1)}
            & BS
				& SB
            & {\footnotesize \it (1)}
            & EN
            & A
            & S
        \\\cline{2-12}

            & \texttt{Control}
            & W
            & \multicolumn{6}{c|}{\footnotesize \it (1)}
				& EN
            & \multicolumn{2}{c|}{\footnotesize \it (1)}
        \\\hline
                \multicolumn{11}{l}{}
        \\
                \multicolumn{11}{l}{\footnotesize \it{Notes: }}
        \\
                \multicolumn{11}{l}{\footnotesize{(1) Reserved. Read values are undefined. Write zero.}}
    \end{tabular}
		\caption{Video DMA Controller}
		\label{tab:video_dma}
\end{table}

The incoming video is stored to memory, starting at the address specified in the {\it Buffer} register. The 
{\it BackBuffer} register is used to store an alternate memory location. To change where the video is stored, 
the new location should first be written into the {\it BackBuffer}. Then the value in the {\it BackBuffer} and
{\it Buffer} registers can be switched by performing a write to the {\it Buffer} register. 

Bit 2 of the {\it Status/Control} register, {\it EN}, is used to enable or disable the Video DMA controller.
In the \systemName, the DMA controller is disabled by default. To enable the DMA controller, write a 
1 into this location. The Video DMA Controller will then begin storing the video into the location specified 
in the {\it Buffer} register.

The default value stored in the {\it Buffer} register is {\sf 0x08000000}. This address is also used as the 
source for the Video-Out port, as described in Section~\ref{sec:video_out}, allowing the Video In stream to be 
displayed on the VGA. If the Video-Out is intended to display a different signal, than the address stored in 
the Video DMA Controller's {\it Buffer} register should be changed. 