\section{Modifying the \systemNameFull}

It is possible to modify the {\it \systemNameFull} 
by using the Quartus\textsuperscript{\textregistered} Prime software
and {\systemBuilder} tool. Instructions for using this software are
provided as part of the
\texttt{Computer Organization and System Design} tutorials on the 
{\small \href{https://www.fpgacademy.org/tutorials.html} {FPGAcademy.org}} website.
To modify the system it is first necessary to make an editable copy of the 
{\it \systemNameFull}. The files for this system are installed as part of the Monitor 
Program installation. Locate these files, copy them to a working directory, and then 
use the Quartus Prime and {\systemBuilder} software to make any desired changes.

Table \ref{tab:sopcnames} lists the names of the {\systemBuilder} IP cores that are used 
in this system. When the {\it \systemNameFull} design files are opened in the Quartus 
Prime software, these cores can be examined using the {\systemBuilder} System Integration 
tool.  Each core has a number of settings that are selectable in the {\systemBuilder} 
System Integration tool, and includes a datasheet that provides detailed documentation.

The steps needed to modify the system are:

\begin{enumerate}
\item Make of copy of the design source files for the {\it \systemNameFull} from the its 
GitHub repository. 
\item Open the top-level project file (*.{\it qpf}) in the Quartus Prime software
\item Open the {\systemBuilder} System Integration tool in the Quartus Prime software, and 
modify the system as desired
\item Generate the modified system by using the {\systemBuilder} System Integration tool
\item It may be necessary to modify the Verilog code in the top-level module of the project, 
if any I/O peripherals have been added or removed from the system
\item Compile the project in the Quartus Prime software
\item Download the modified system into the \DEBoard~board
\end{enumerate}

Note: to compile and use a new version of the {\it \systemNameFull} it may be necessary to
request a license from Altera that allows you to create circuit that includes 
the {\processor} processor.

