\subsection{Using Interrupts with Assembly Language Code}

An example of assembly language code for the \systemName~that uses interrupts is
shown in Listing \ref{lst:interrupt_example_s}. When this code is executed on the \DEBoard~board
it displays a rotating pattern on the LEDs. The pattern's rotation can be toggled through pressing the pushbutton KEYs. Two types
of interrupts are used in the code. The LEDs are controlled by an interrupt
service routine for the interval timer, and another interrupt service routine is used to
handle the pushbutton keys. The speed of the rotation is set in the main
program, by using a counter value in the interval timer that causes an interrupt to occur
every 50 msec.

The reset and exception handlers for the main program in Listing \ref{lst:interrupt_example_s} 
are given in Listing \ref{lst:exception_handler_s}. The reset handler simply jumps to the
{\it \_start} symbol in the main program. The exception handler first checks if the
exception that has occurred is an external interrupt or an internal one. In the case of an
internal exception, such as an illegal instruction opcode or a trap instruction, the
handler simply exits, because it does not handle these cases. For external exceptions, it 
calls either the interval timer interrupt service routine, for a level 0 interrupt, or the
pushbutton key interrupt service routine for level 1. These routines are shown in Listings
\ref{lst:interval_timer_isr_s} and \ref{lst:pushbutton_isr_s}, respectively.


