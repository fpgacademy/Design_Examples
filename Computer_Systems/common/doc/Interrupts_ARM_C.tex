\subsection{Using Interrupts with C Code}

An example of C code for the \systemName~that uses interrupts is
shown in Figure \ref{lst:interrupt_example_C}. This code performs exactly the same
operations as the code described in Listing~\ref{lst:interrupt_example_s}. 

Before it call subroutines to configure the generic interrupt controller (GIC), timers,
and pushbutton KEY port, the main program first initializes the IRQ mode stack pointer by
calling the routine {\it set\_A9\_IRQ\_stack()}. The code for this routine uses in-line
assembly language instructions, as shown in Part $b$ of the listing. This step is necessary 
because the C compiler generates code to set only the supervisor mode stack, which is used 
for running the main program, but the compiler does not produce code for setting
the IRQ mode stack.  To enable IRQ interrupts in the A9 processor the main program uses
the in-line assembly code shown in the subroutine called {\it enable\_A9\_interrupts()}.

The exception handlers for the main program in Listing~\ref{lst:interrupt_example_C} 
are given in Listing~\ref{lst:exception_handler_C}. These routines have unique names that
are meaningful to the C compiler and linker tools, and they are declared with the special
type of {\bf \_\_attribute\_\_} called {\bf interrupt}. These mechanisms cause the C
compiler and linker to use the addresses of these routines as the contents of the exception
vector table.

The function with the name {\it \_\_cs3\_isr\_irq} is the IRQ exception handler. As
discussed for the assembly language code in Listing~\ref{lst:exception_handler_s} this 
routine first reads from the {\it interrupt acknowledge} register in the GIC to determine 
the interrupt ID of the peripheral that caused the interrupt, and then 
calls the corresponding interrupt service routine for either the HPS timer, FPGA interval timer,
or FPGA KEY parallel port.  These interrupt service routines are shown in 
Listings~\ref{lst:hps_timer_isr_C} to \ref{lst:interval_timer_isr_C}.

Listing~\ref{lst:exception_handler_C} also shows handlers for exceptions that correspond to 
the various types of error conditions and the FIQ interrupt. These handlers are just loops
that are meant to serve as place-holders for code that would handle the corresponding
exceptions. 

The source code files shown in Listing~\ref{lst:interrupt_example_s} to
Listing~\ref{lst:pushbutton_isr_C} are distributed as part of the  
\productNameMed{}. The files can be found under the heading {\it sample programs}, 
and are identified by the name {\it Interrupt Example}.


