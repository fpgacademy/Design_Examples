\begin{figure}[h!]
\begin{center}
\begin{minipage}[t]{12.5 cm}
\begin{tabbing}
/\=*****\=*********************************\=****************************************\=\kill
/********************************************************************************\\
\>* This program demonstrates use of the JTAG UART port in the DE10-Lite Computer\\
\>*\\
\>* It performs the following: \\
\>* \>1. sends a text string to the JTAG UART\\
\>* \>2. reads character data from the JTAG UART\\
\>* \>3. echos the character data back to the JTAG UART\\
\=\kill
\>********************************************************************************/\\
ZZZZZZZZ\={\bf movia}ZZZ\=r16, 0xFF200000ZZZZZ\=/* RED\_LED base address */\kill
\>.{\bf text}	\>\>/* executable code follows */\\
\>.{\bf global}\>\_start\\
\_start:\\
\>/* set up stack pointer */\\
\>{\bf movia} \>sp, 0x03FFFFFC \>/* put stack at top of SDRAM */\\
\rule{6.0in}{0in}~\\
\>{\bf movia} \>r6, 0xFF201000 \>/* JTAG UART base address */\\
\\
\>/* print a text string */\\
\>{\bf movia} \>r8, TEXT\_STRING\\
LOOP:\\
\>{\bf ldb} \>r5, 0(r8)\\
\>{\bf beq} \>r5, zero, GET\_JTAG \>/* string is null-terminated */\\
\>{\bf call} \>PUT\_JTAG\\
\>{\bf addi} \>r8, r8, 1\\
\>{\bf br} \>LOOP\\
~\\
\>/* read and echo characters */\\
GET\_JTAG:\\
\>{\bf ldwio} \>r4, 0(r6) \>/* read the JTAG UART Data register */\\
\>{\bf andi} \>r8, r4, 0x8000 \>/* check if there is new data */\\
\>{\bf beq} \>r8, r0, GET\_JTAG \>/* if no data, wait */\\
\>{\bf andi} \>r5, r4, 0x00ff \>/* the data is in the least significant byte */\\
~\\
\>{\bf call} \>PUT\_JTAG \>/* echo character */\\
\>{\bf br} \>GET\_JTAG\\
\>.{\bf end}\\
\end{tabbing}
\end{minipage}
\end{center}
	\vspace{-0.33in}\caption{An example of assembly language code that uses the JTAG UART
	(Part {\it a}).}
   \label{fig:jtag_uart_s}
\end{figure}
\pagebreak
\clearpage
\newpage

\begin{center}
\begin{minipage}[t]{12.5 cm}
\begin{tabbing}
/\=*****\=***\=******************************\=****************************************\=\kill
/********************************************************************************\\
\>* Subroutine to send a character to the JTAG UART\\
\>* \>r5 \>= character to send\\
\>* \>r6 \>= JTAG UART base address\\
\=\kill
\>********************************************************************************/\\
ZZZZZZZZ\={\bf movia}ZZZ\=r16, 0xFF200000ZZZZZ\=/* RED\_LED base address */\kill
\>.{\bf global} \>PUT\_JTAG\\
PUT\_JTAG:\\
\>/* save any modified registers */\\
\>{\bf subi} \>sp, sp, 4 \>/* reserve space on the stack */\\
\>{\bf stw} \>r4, 0(sp) \>/* save register */\\
\\
\>{\bf ldwio} \>r4, 4(r6) \>/* read the JTAG UART Control register */\\
\>{\bf andhi} \>r4, r4, 0xffff \>/* check for write space */\\
\>{\bf beq} \>r4, r0, END\_PUT \>/* if no space, ignore the character */\\
\>{\bf stwio} \>r5, 0(r6) \>/* send the character */\\
\rule{6.0in}{0in}~\\
END\_PUT:\\
\>/* restore registers */\\
\>{\bf ldw} \>r4, 0(sp)\\
\>{\bf addi} \>sp, sp, 4\\
\\
\>{\bf ret}\\
\\
\>.{\bf data} \>\>/* data follows */\\
TEXT\_STRING:\\
\>.{\bf asciz} "$\backslash$nJTAG UART example code$\backslash$n$>$ "\\
\\
\>.{\bf end}\\
~\\
Figure \ref{fig:jtag_uart_s}. An example of assembly language code that uses the JTAG
UART (Part {\it b}).
\end{tabbing}
\end{minipage}
\end{center}