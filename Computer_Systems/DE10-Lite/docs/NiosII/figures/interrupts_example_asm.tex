\begin{figure}[h!]
\begin{center}
\begin{minipage}[t]{12.5 cm}
\begin{tabbing}
ZZ\={\bf movia}ZZZ\=r16, 0xFF200000ZZZZZ\=/* RED\_LED base address */\kill
\>.{\bf equ} \>KEY, 1\\
/\=*****\=*********************************\=****************************************\=\kill
/********************************************************************************\\
\>* This program demonstrates use of interrupts in the DE10-Lite Computer. It first starts the \\
\>* interval timer with 33 msec timeouts, and then enables interrupts from the interval
timer\\
\>* and pushbutton KEYs\\
\>*\\
\>* The interrupt service routine for the interval timer displays a pattern on the HEX3-0\\
\>* displays, and rotates this pattern either left or right:\\
\>* ~~~~~KEY[0]: loads a new pattern from the SW switches\\
\>* ~~~~~KEY[1]: toggles the direction of the rotation\\
\=\kill
\>********************************************************************************/\\
ZZ\={\bf movia}ZZZ\=r16, 0xFF200000ZZZZZ\=/* RED\_LED base address */\kill
\>.{\bf text}	\>\>/* executable code follows */\\
\>.{\bf global}\>\_start\\
\_start:\\
\>/* set up stack pointer */\\
\>{\bf movia} \>sp, 0x03FFFFFC \>/* stack starts from highest memory address in SDRAM */\\
\rule{6.0in}{0in}~\\
\>{\bf movia} \>r16, 0xFF202000 \>/* internal timer base address */\\
\>/* set the interval timer period for scrolling the HEX displays */\\
\>{\bf movia} \>r12, 5000000 \>/* 1/(100 MHz) $\times$ (5000000) = 50 msec */\\
\>{\bf sthio} \>r12, 8(r16) \>/* store the low halfword of counter start value */ \\
\>{\bf srli} \>r12, r12, 16\\
\>{\bf sthio} \>r12, 0xC(r16) \>/* high halfword of counter start value */ \\
\end{tabbing}
\end{minipage}
\end{center}
	\vspace{-0.33in}\caption{An example of assembly language code that uses interrupts
	(Part {\it a}).}
   \label{fig:interrupt_example_s}
\end{figure}
\pagebreak
\clearpage
\newpage

\begin{center}
\begin{minipage}[t]{12.5 cm}
\begin{tabbing}
/\=*****\=*********************************\=****************************************\=\kill
ZZ\={\bf movia}ZZZ\=r16, 0xFF200000ZZZZZ\=/* RED\_LED base address */\kill
\>/* start interval timer, enable its interrupts */\\
\>{\bf movi} \>r15, 0b0111 \>/* START = 1, CONT = 1, ITO = 1 */\\
\>{\bf sthio} \>r15, 4(r16)\\
\rule{6.0in}{0in}~\\
\>/* write to the pushbutton port interrupt mask register */\\
\>{\bf movia} \>r15, 0xFF200050 \>/* pushbutton key base address */\\
\>{\bf movi} \>r7, 0b1111 \>/* set interrupt mask bits */\\
\>{\bf stwio} \>r7, 8(r15) \>/* interrupt mask register is (base + 8) */\\
~\\
\>/* enable Nios II processor interrupts */\\
\>{\bf movi} \>r7, 0b011 \>/* set interrupt mask bits for levels 0 (interval */\\
\>{\bf wrctl} \>ienable, r7 \>/* timer) and level 1 (pushbuttons) */\\
\>{\bf movi} \>r7, 1\\
\>{\bf wrctl} \>status, r7 \>/* turn on Nios II interrupt processing */\\
~\\
IDLE:\\
\>{\bf br} \>IDLE \>/* main program simply idles */\\
~\\
\>.{\bf data}\\
/\=* The two global variables \kill
/* The global variables used by the interrupt service routines for the interval timer and the\\
\>* pushbutton keys are declared below */\\
~\\
ZZ\={\bf movia}ZZZ\=r16, 0xFF200000ZZZZZ\=/* RED\_LED base address */\kill
\>.{\bf global} \>PATTERN\\
PATTERN:\\
\>.{\bf word} \>0x0000000F \>/* pattern to show on the HEX displays */\\
~\\
\>.{\bf global} \>KEY\_PRESSED\\
KEY\_PRESSED:\\
\>.{\bf word} \>KEY \>/* stores code representing pushbutton key pressed */\\
~\\
\>.{\bf global} \>SHIFT\_DIR\\
SHIFT\_DIR:\\
\>.{\bf word} \>2 \>/* default shift direction (2 == right) */\\
~\\
\>.{\bf end}
~\\
~\\
Figure \ref{fig:interrupt_example_s}. An example of assembly language code that uses interrupts (Part {\it b}).
\end{tabbing}
\end{minipage}
\end{center}
