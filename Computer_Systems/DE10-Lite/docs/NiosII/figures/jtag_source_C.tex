\begin{figure}[h!]
\begin{center}
\begin{minipage}[t]{12.5 cm}
\begin{tabbing}
ZZ\=ZZ\=ZZ\=HEX\_bits = HEX\_bits z 0xFFFFFFFF;ZZ\=// invert pattern for HEX displays\kill
/* function prototypes */\\
{\bf void} put\_jtag({\bf char});\\
{\bf char} get\_jtag({\bf void});\\
/\=*****\=*********************************\=****************************************\=\kill
/********************************************************************************\\
\>* This program demonstrates use of the JTAG UART port in the DE10-Lite Computer\\
\>* It performs the following: \\
\>* \>1. sends a text string to the JTAG UART\\
\>* \>2. reads and echos character data from/to the JTAG UART\\
\=\kill
\>********************************************************************************/\\
{\bf int} main({\bf void})\\
\{\\
ZZ\=ZZ\=ZZ\=HEX\_bits = HEX\_bits z 0xFFFFFFFF;ZZ\=// invert pattern for HEX displays\kill
\>{\bf char} text\_string[$\,$] = "$\backslash$nJTAG UART example code$\backslash$n$>$ $\backslash$0";\\
\>{\bf char} *str, c;\\
~\\
\>/* print a text string */\\
\>{\bf for} (str = text\_string; *str != 0; ++str)\\
\>\>put\_jtag (*str);
~\\
\>/* read and echo characters */\\
\>{\bf while} (1)\\
\>\{\\
\>\>c = get\_jtag ( );\\
\>\>{\bf if} (c != '$\backslash$0')\\
\>\>\>put\_jtag (c);\\
\>\}\\
\}
\rule{6.0in}{0in}~\\\\
/\=*****\=*********************************\=****************************************\=\kill
/********************************************************************************\\
\>* Subroutine to send a character to the JTAG UART\\
\=\kill
\>********************************************************************************/\\
ZZ\=ZZZ\=ZZZ\=HEX\_bits = HEX\_bits z ;ZZ\=// invert pattern for HEX displays\kill
{\bf void} put\_jtag( {\bf char} c )\\
\{\\
\>{\bf volatile int} * JTAG\_UART\_ptr 	= (int *) 0xFF201000;  // JTAG UART address\\
\>{\bf int} control;\\
\>control = *(JTAG\_UART\_ptr + 1);	\>\>\>// read the JTAG\_UART control register\\
\>{\bf if} (control \& 0xFFFF0000) \>\>\>// if space, echo character, else ignore \\
\>\>*(JTAG\_UART\_ptr) = c;\\
\}\\
\end{tabbing}
\end{minipage}
\end{center}
	\vspace{-0.33in}\caption{An example of C code that uses the JTAG UART (Part {\it a}).}
   \label{fig:jtag_uart_C}
\end{figure}
\pagebreak
\clearpage
\newpage
\begin{figure}[h!]
\begin{center}
\begin{minipage}[t]{12.5 cm}
\begin{tabbing}
/\=*****\=*********************************\=****************************************\=\kill
/********************************************************************************\\
\>* Subroutine to read a character from the JTAG UART\\
\>* Returns $\backslash$0 if no character, otherwise returns the character\\
\=\kill
\>********************************************************************************/\\
ZZ\=ZZZ\=ZZZ\=HEX\_bits = HEX\_bits z ;ZZ\=// invert pattern for HEX displays\kill
{\bf char} get\_jtag( {\bf void} )\\
\{\\
\>{\bf volatile int} * JTAG\_UART\_ptr	= ({\bf int} *) 0xFF201000;	// JTAG UART address\\
\>{\bf int} data;\\
\>data = *(JTAG\_UART\_ptr); \>\>\>// read the JTAG\_UART data register\\
\>{\bf if} (data \& 0x00008000) \>\>\>// check RVALID to see if there is new data\\
\>\>{\bf return} (({\bf char}) data \& 0xFF);\\
\>{\bf else}\\
\>\>{\bf return} ('$\backslash$0');\\
\}
\end{tabbing}
\end{minipage}
\end{center}
	\vspace{-0.33in}\caption{An example of C code that uses the JTAG UART (Part {\it b}).}
   \label{fig:jtag_uart_C}
\end{figure}