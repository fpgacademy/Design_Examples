\begin{figure}[h!]
\begin{center}
\begin{minipage}[t]{12.5 cm}
\begin{tabbing}
\rule{6.0in}{0in}~\\
/\=*****\=*********************************\=****************************************\=\kill
/********************************************************************************\\
\>* This program demonstrates use of floating-point numbers in the DE10-Lite Computer\\
\>*\\
\>* It performs the following: \\
\>* 	\>1. reads two FP numbers from the Terminal window\\
\>* 	\>2. performs +, 1, *, and / on the numbers, then prints results on Terminal window\\
\=\kill
\>********************************************************************************/\\
ZZZ\=ZZZ\=printf ("\%f ", x);ZZZ\=// echo the typed data to the Terminal window\kill
{\bf int} main({\bf void})\\
\{\\
\>{\bf float} x, y, add, sub, mult, div;\\
~\\
\>{\bf while}(1)\\
\>\{\\
\>\>printf ("Enter FP values X Y:$\backslash$n");\\
\>\>scanf ("\%f", \&x);\\
\>\>printf ("\%f ", x);		// echo the typed data to the Terminal window\\
\>\>scanf ("\%f", \&y);\\
\>\>printf ("\%f$\backslash$n", y);		// echo the typed data to the Terminal window\\
\>\>add = x + y;\\
\>\>sub = x $-$ y;\\
\>\>mult = x * y;\\
\>\>div = x / y;\\
\>\>printf ("X + Y = \%f$\backslash$n", add);\\
\>\>printf ("X - Y = \%f$\backslash$n", sub);\\
\>\>printf ("X * Y = \%f$\backslash$n", mult);\\
\>\>printf ("X / Y = \%f$\backslash$n", div);\\
\>\}\\
\}
\end{tabbing}
\end{minipage}
\end{center}
	\vspace{-0.33in}\caption{An example of code that uses floating-point variables.}
   \label{fig:fp}
\end{figure}