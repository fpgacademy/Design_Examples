\begin{figure}[h!]
\begin{center}
\begin{minipage}[t]{12.5 cm}
\begin{tabbing}
{\bf volatile}	{\bf int} pattern = 0x0000000F;ZZZ\=// pattern for HEX displays\kill
\#{\bf include} "nios2\_ctrl\_reg\_macros.h"\\
\#{\bf include} "key\_codes.h"\>// defines values for KEY0, KEY1, $\ldots$\\
~\\
/\=* \kill
/* key\_pressed and pattern are written by interrupt service routines; we have to declare \\
\>* these as volatile to avoid the compiler caching their values in registers */\\
{\bf volatile}	{\bf int} pattern = 0x0000000F;ZZZ\=// pattern for HEX displays\kill
{\bf volatile} {\bf int} key\_pressed = KEY1;\>// shows which key was last pressed\\
{\bf volatile}	{\bf int} pattern = 0x0000000F;\>// pattern for HEX displays\\
{\bf volatile}	{\bf int} shift\_dir = LEFT;	\>// direction to shift the pattern\\
/\=*****\=*********************************\=****************************************\=\kill
/********************************************************************************\\
\>* This program demonstrates use of interrupts in the DE10-Lite Computer. It first starts the \\
\>* interval timer with 33 msec timeouts, and then enables interrupts from the interval
timer\\
\>* and pushbutton KEYs\\
\>*\\
\>* The interrupt service routine for the interval timer displays a pattern on the HEX3-0\\
\>* displays, and rotates this pattern either left or right:\\
\>* ~~~~~KEY[0]: loads a new pattern from the SW switches\\
\>* ~~~~~KEY[1]: rotates the displayed pattern to the right\\
\>* ~~~~~KEY[2]: rotates the displayed pattern to the left\\
\>* ~~~~~KEY[3]: stops the rotation\\
\=\kill
\>********************************************************************************/\\
{\bf int} main({\bf void})\\
\{\\
ZZ\=/\=* Declare volatile pointers to I/O registers (volatile means that IO load\kill
\>/* Declare volatile pointers to I/O registers (volatile means that IO load and store instructions\\
\>\>* will be used to access these pointer locations instead of regular memory loads and stores) */\\
ZZ\={\bf volatile} {\bf int} * interval\_timer\_ptr = ({\bf int} *) 0xFF202000;ZZ\=// interval timer base address\kill
\>{\bf volatile} {\bf int} * interval\_timer\_ptr = ({\bf int} *) 0xFF202000;\>// interval timer base address\\
\>{\bf volatile} {\bf int} * KEY\_ptr = ({\bf int} *) 0xFF200050;\>// pushbutton KEY address\\
ZZ\=NIOS2\_WRITE\_IENABLE( 0x3 );ZZZ\=/\=* set interrupt mask bits for levels 0 (interval\kill
\rule{6.0in}{0in}~\\
\>/* set the interval timer period for scrolling the HEX displays */\\
\>{\bf int} counter = 5000000; \>// 1/(100 MHz) $\times$ (5000000) = 50 msec\\
\>*(interval\_timer\_ptr + 0x2) = (counter \& 0xFFFF);\\
\>*(interval\_timer\_ptr + 0x3) = (counter $>>$ 16) \& 0xFFFF;\\
~\\
\>/* start interval timer, enable its interrupts */\\
\>*(interval\_timer\_ptr + 1) = 0x7;\>// STOP = 0, START = 1, CONT = 1, ITO = 1 \\
~\\
\>*(KEY\_ptr + 2) = 0xF; \>/* write to the pushbutton interrupt mask register, and\\
\>\>\>* set mask bits to 1 */\\
~\\
\>NIOS2\_WRITE\_IENABLE( 0x3 ); \>/* set interrupt mask bits for levels 0 (interval timer)\\
\>\>* and level 1 (pushbuttons) */\\
\>NIOS2\_WRITE\_STATUS( 1 ); \>// enable Nios II interrupts\\
~\\
\>{\bf while}(1); \>// main program simply idles\\
\}\\
\end{tabbing}
\end{minipage}
\end{center}
	\vspace{-0.33in}\caption{An example of C code that uses interrupts.}
   \label{fig:interrupt_example_C}
\end{figure}
