\begin{figure}[h!]
\begin{center}
\begin{minipage}[t]{12.5 cm}
\begin{tabbing}
ZZZ\=ZZ\=ZZ\=*(pixel\_buffer + offset) = pixel\_color;ZZZZZZZ\= \kill
{\bf volatile} {\bf short} * pixel\_buffer = ({\bf short} *) 0x08000000;\>\>\>\>// Pixel buffer\\
{\bf volatile} {\bf char} * character\_buffer = ({\bf char} *) 0x09000000;\>\>\>\>// Character buffer\\
{\bf int} x1, {\bf int} y1, {\bf int} x2, {\bf int} y2, {\bf short} pixel\_color;\\
{\bf int} offset, row, col;\\
{\bf int} x, {\bf int} y, {\bf char} * text\_ptr;\\
$\ldots$\\
/* Draw a box; assume that the coordinates are valid */\\
{\bf for} (row = y1; row $<$= y2; row++)\\
ZZZ\=ZZ\=ZZ\=*(pixel\_buffer + offset) = pixel\_color;ZZZZZZZ\= \kill
\>{\bf for} (col = x1; col <= x2; ++col)\\
\>\{\\
\>\>offset = (row $<<$ 9) + col;\\
\>\>*(pixel\_buffer + offset) = pixel\_color;	\>\>// compute halfword address, set pixel\\
\>\}\\
\\
/* Display a text string; assume that it fits on one line */\\
offset = (y $<<$ 7) + x;\\
{\bf while} ( *(text\_ptr) )\\
\{\\
\>*(character\_buffer + offset) = *(text\_ptr);	\>\>\>// write to the character buffer\\
\>++text\_ptr;\\
\>++offset;\\
\}\rule{6.0in}{0in}~\\
\end{tabbing}
\end{minipage}
\end{center}
	\vspace{-0.33in}\caption{An example of code that uses the video-out port.}
   \label{fig:video_C}
\end{figure}
